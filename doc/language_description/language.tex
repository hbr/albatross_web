\chapter{Language Reference}

\section{Structure of Albatross Programs}

Modules are compilation units. Each module has an implementation file with the
extension \lstinline!.al! and an interface file with the extension
\lstinline!.ali!. The user of a module has access only to the elements
declared in the interface file.

A collection of modules in the same directory form an Albatross package or an
Albatross library. The compiler works on Albatross packages. Within a
directory of Albatross source files the compiler offers you the following
basic commands:

\begin{lstlisting}
    alba init       -- initializes the directory
    alba status     -- displays modules which need recompilation
    alba compile    -- compiles all modules which require recompilation
    alba help       -- get help about commands, options and arguments
\end{lstlisting}
  
I.e. the albatross compiler automatically keeps track of dependencies between
modules.

Each module has to start with a declaration which states the used modules.

\begin{lstlisting}
    -- some source file
    use
        module_1
        module_2
        ...
    end
    
    ...
\end{lstlisting}

The order of the used modules is irrelevant. The modules used by \lstinline!module_1!
and \lstinline!module_2! are used implicitly.

The name of a module is the filename of the module's source file without its
extension. I.e. the module name of the module having the source files
\lstinline!boolean.al! and \lstinline!boolean.ali! is \lstinline!boolean!.

Modules of the same package are used by name. Modules of other packages are
used by their fully qualified names. E.g. a module with the usage declaration
\begin{lstlisting}
    use
        alba.base.boolean
        alba.base.predicate
        graphic
    end
\end{lstlisting}
uses the modules \lstinline!boolean! and \lstinline!predicate! of the package
\lstinline!alba.base! and the module \lstinline!graphic! of the same package.

The package name is the name of the directory where it resides. The package
\lstinline!alba.base! has to reside in a directory named \lstinline!alba.base!.

The Albatross compiler searches for packages within its search path. Search
paths can be given to the compiler on the command line
\begin{lstlisting}
    alba -I path/to/package_1 -I path/to/package_2 ... compile
\end{lstlisting}
or by defining an environment variable called \lstinline!ALBA_LIB_PATH!. If the
environment variable has the content
%
\begin{lstlisting}
    path/to/lib_1:path/to/lib_2:path/to/lib_3:...
\end{lstlisting}
%
then the Albatross compiler searches for libraries in all these paths.

The paths on the command line override the paths of the environment variable.

The locations of commonly used libraries should be given by the environment
variable and the location of libraries/packages used only for a specific
program should be given on the command line.



\section{Modules}

A module is the basic compilation unit. Each module has an implementation file
with the name \lstinline!name.al! an an optional interface file with the name
\lstinline!name.ali!  where \lstinline!name! is the name of the module.

The interface file just defines the public view of a module. Therefore it
cannot introduce anything which does not appear in the corresponding
implementation file.

A module without an interface file cannot be used by other modules.

A module file consists of a usage header and a sequence of declarations.

\noindent Syntax:
\begin{lstlisting}
    module:       [use_block] declarations

    use_block:    use module_list end

    module_list:  {one_module separator ...}+

    one_module:   {identifier '.' ...}+

    declarations: {declaration [separator] ...}*

    declaration:  class_declaration
    |             formal_generic
    |             assertion_feature
    |             named_feature

    formal_generic:     uidentifier ':' type
\end{lstlisting}

\noindent Rules and validity:
\begin{enumerate}
\item No circularity is allowed in the usage of modules.

\item An interface file can use only modules which have been used in the
implementation file. Usage can be implicit. E.g. if module \lstinline!a! uses
module \lstinline!b!  and module \lstinline!b! uses module \lstinline!c! then
module \lstinline!a! uses module \lstinline!c! implicitly even if module
\lstinline!c! has not been mentioned in the usage block of module
\lstinline!a!.

\item A used module of the same package must not be prefixed with the package
  name.
\end{enumerate}

A separator can be a semicolon or a newline (for details on newlines as
separators see the chapter lexical conventions \ref{lexconv}).

A used module in the usage block has the form \lstinline!a.b.c.m! where
\lstinline!m! is the name of the module and \lstinline!a.b.c! is the name of
the package where the module resides.

A formal generic is a type variable. The declaration \lstinline!A:TP! declares
the type variable \lstinline!A! with the concept type \lstinline!TP!. The type
variable \lstinline!A! can be replaced by any type which satisfies the concept
(i.e. inherits from) \lstinline!TP!.

The name scope of a formal generic is its module.

\noindent Examples:

The module \lstinline!boolean! of the basic library e.g. has the following
interface file:

\begin{lstlisting}
    -- file: boolean.ali

    immutable class BOOLEAN end

    false: BOOLEAN
    true:  BOOLEAN
    (==>) (a,b:BOOLEAN): BOOLEAN
    (and) (a,b:BOOLEAN): BOOLEAN
    (or)  (a,b:BOOLEAN): BOOLEAN

    (not) (a:BOOLEAN): BOOLEAN)   -> a ==> false
    (=)   (a,b:BOOLEAN): BOOLEAN) -> (a ==> b) and (b ==> a)
    
    all(a,b:BOOLEAN)
        ensure
            (not a ==> false) ==> a
            a and b ==> a
            a and b ==> b
            ...
        end
\end{lstlisting}

The module declares the type \lstinline!BOOLEAN! as immutable. I.e. all
objects of type \lstinline!BOOLEAN! cannot be modified.

If a class has the same name as the module, just in uppercase, then the class
is available to the users without any qualification. If a class has a name
different from the module name it can be accessed by the users of the module
only by qualifying the class name with the modulename. E.g. the class
\lstinline!A! declared in the module \lstinline!x! has to be used outside the
defining module as \lstinline!x.A!. Sometimes it might even be necessary to
add the package name (e.g. \lstinline!p.x.A!) to disambiguate the situation.

Furthermore it defines the constants \lstinline!true! and
\lstinline!false!. No definition of these constants are visible to the
user. The implementation file might have a definition of these constants, but
they are hidden from the user.

The three binary boolean functions implication \lstinline!==>!, conjunction
\lstinline!and! and disjunction \lstinline!or! are declared without definition
which means their definition is not available to the user.

The functions negation \lstinline!not! and boolean equivalence \lstinline!=!
are declared with a definition. Since the definition is visible the compiler
can expand any expression of the form \lstinline!not (x + y = y)! to
\lstinline!(x + y = z) ==> false!

The assertion declaration states valid assertions about booleans. In the
interface file the assertions are just stated. The proofs can be found in the
corresponding implementation file \lstinline!boolean.al!.

The first assertion \lstinline!(not a ==> false) ==> a! gives us the
opportunity to prove any assertion \lstinline!a! by assuming \lstinline!not a!
and deriving from it \lstinline!false!.


The module \lstinline!any! of the basic library has the following interface file:

\begin{lstlisting}
    -- file: any.ali
    use boolean end

    deferred class ANY end

    G: ANY

    (=)  (a,b:G): BOOLEAN    deferred end
    (/=) (a,b:G): BOOLEAN -> not (a = b)

    all(a:G) deferred ensure a = a end
    all(a:G) ensure a /= a ==> false end

    immutable class boolean.BOOLEAN
    inherit         ANY end
\end{lstlisting}

The module declares the class \lstinline!ANY! as a deferred class which means
that it can have deferred functions and/or assertions which have to be defined
in other classes which inherit from the class \lstinline!ANY!.

The module defines a formal generic \lstinline!G! which is a type
variable. Any type can be substituted for \lstinline!G! as long as the type
inherits from the class \lstinline!ANY!.

The module \lstinline!any! has the purpose to define equality. Therefore it
declares a deferred function named \lstinline!=! which compares two objects of
the same type. Any descendant of \lstinline!ANY! has to define the function or
redeclare it as deferred. Furthermore it declares an assertion stating
reflexivity of equality. Since the assertion is deferred any descendant of the
class \lstinline!ANY!  has to either prove this statement or restate it as
deferred.

Inequality \lstinline!/=! is defined in the module \lstinline!any! as an
effective function. This function is inherited to any descendant.

Furthermore the module \lstinline!any! states that it has proved the fact that
\lstinline!a /= a! \lstinline! ==> false! is valid for any \lstinline!a!. Any
descendant of \lstinline!ANY! can take this fact for granted.

Note that the class name \lstinline!BOOLEAN! is qualified with the module name
in the inheritance class despite the fact that the class \lstinline!BOOLEAN!
has the same name as the corresponding module. The qualification is necessary
in this case because otherwise the compiler would generate a new class
\lstinline!BOOLEAN! of the module \lstinline!any! which is different from the
corresponding class of the module \lstinline!boolean!.

The module \lstinline!any! states with this inheritance clause that the class
\lstinline!BOOLEAN! can inherit the class \lstinline!ANY! i.e. that it has a
function \lstinline!=! which is reflexive.

The module \lstinline!boolean! cannot state this inheritance relation because
it cannot use the module \lstinline!any! since the module \lstinline!any!
already uses the module \lstinline!boolean!  and no circularity is allowed.



\section{Classes}
%
\noindent Syntax:
\begin{lstlisting}
    class_declaration:  [header_mark] class
                        class_name [class_generics]
                        [creator_clause]
                        [inherit_clause]
                        end

    header_mark:        immutable | deferred | case

    class_name:         {identifier '.' ...}* uidentifier

    class_generics:     '[' {uidentifier ',' ...}+ ']'
\end{lstlisting}


\noindent Rules and validity:
\begin{enumerate}
\item In a class declaration only formal generics can be used which have
  already been declared in the surrounding module.

\item A redeclaration of a class must have the same header mark, the same
  number of generics and the formal generics must have the same concepts as
  the previous declaration.

\item All deferred functions and assertions of a class have to be declared in
  the same module as the first declaration of the class.

\item If a class has already been used as a parent in an inheritance clause
  then no more deferred functions and assertions can be declared for the class.
\end{enumerate}

The first declaration of a class in a module does not have an inheritance
clause i.e. it has one of the forms

\begin{lstlisting}
    immutable class BOOLEAN end        -- file: boolean.al[i]

    deferred class ANY end             -- file: any.al[i]

    deferred class PARTIAL_ORDER end   -- file: partial_order.al[i]

    immutable class FUNCTION[A,B] end  -- file: function.al[i]
\end{lstlisting}
After the class has been declared with its potential formal generics, features
and assertions can be introduced to the class.

Classes may add inheritance clauses if they also declare the appropriate
features and assertions.

A class declaration where the classname is unqualified always declares a class
which belongs to the current module. The first declaration of a class within a
module introduces a new class.

If a class has the same name as the module (just in upper case) then the class
can be used by other modules to declare types without any qualification
(however qualification can be used to remove ambiguities).

A class having a name different from the module name can be used outside the
defining module to define types only by adding the module name to the class
name in front of the class name.

A class can redeclared as often as needed (usually to add inheritance clauses)
as long as the redeclaration is consistent with the original declaration.

Classes can be redeclared in modules different from the original module, but
in the reclaration the class name has to be prefixed with the name of the
original module. Without the module name a new class having
the same name is declared. The following example illustrates this.

\begin{lstlisting}
    -- file a.ali
    immutable class A end   -- declares a new class 'A' in module 'a'

    -- file b.ali
    immutable class A end   -- declares a new class 'A' in module 'b'

    immutable class a.A end -- redeclares the class 'A' of the module 'a'
\end{lstlisting}

\section{Creators}

\noindent Syntax:
\begin{lstlisting}
    creator_clause:    create  { creator [separator] ...}+

    creator:           feature_name ['(' formals ')' ]

\end{lstlisting}

\noindent Definition: A base creator is a creator which has no argument
depending on the surrounding case class. The other creators are recursive.

\noindent Rules and validity:
\begin{enumerate}
\item There has to be at least one base creator.

\item All base creators have to occur before the recursive creators.

\item The creators can use only formal generics of the surrounding class.

\item All formal generics have to appear in at least one of the constructors
  in a nonrecursive argument.
\end{enumerate}


\section{Inheritance}

\noindent Syntax:
\begin{lstlisting}
    inherit_clause:    inherit {parent [separator] ...}+

    parent:            [ghost] type
\end{lstlisting}

\noindent Rules and validity:
\begin{enumerate}
\item Immutable types cannot be used as parents.

\item A parent can involve only formal generics of the inheriting class.

\item The inheritance relation must be acyclic.

\item All deferred functions and assertions of the parent have to be present
  in the inheriting class, either redeclared as deferred or redeclared with a
  definition or proof respectively.

\item All effective features of the parent which have already a declaration in
  the inheriting class must be consistent with the definition in the parent.

\item If a deferred function of the parent is present in the inheriting class
  as a ghost function then the parent must be a ghost type.
\end{enumerate}

\section{Types}

\noindent Syntax:
\begin{lstlisting}
    type:   simple_type
    |       predicate_type
    |       function_type
    |       list_type
    |       '(' {type ',' ...}+ ')'

    simple_type:       class_name [actual_generics]

    actual_generics:   '[' {type ',' ...}+ ']'

    predicate_type:    type '?'

    function_type:     type '->' type

    list_type:         '[' {type ',' ...}+ ']'

    -- '?' has higher precedence than '->'
\end{lstlisting}


\noindent Rules and validity:
\begin{enumerate}
\item The number of actual generics has to coincide with the number of formal
  generics of the class.

\item Each actual generic has to satisfy the concept of the corresponding
  formal generic of the class.
\end{enumerate}

\noindent Examples:
%
\begin{lstlisting}
    (A,B,C)      -- tuple with three types
                 -- parsed as (A,(B,C))
                 -- shorthand for TUPLE[A,TUPLE[B,C]]

    A -> B       -- shorthand for FUNCTION[A,B]

    (A,B) -> C   -- shorthand for FUNCTION[(A,B),C]

    A -> B -> C  -- shorthand for FUNCTION[A, FUNCTION[B,C]]
                 -- '->' is right associative

    A?           -- shorthand for PREDICATE[A] i.e. a set of
                 -- elements of type A

    (A,B)?       -- shorthand for PREDICATE[(A,B)] i.e. a set of
                 -- tuples of type (A,B)

    A -> B?      -- shorthand for FUNCTION[A,PREDICATE[B]]

    (A -> B)?    -- shorthand for PREDICATE[FUNCTION[A,B]]

    [A]          -- shorthand for LIST[A]

    [A,B->C]     -- shorthand for LIST[TUPLE[A,FUNCTION[B,C]]]
\end{lstlisting}

The basic library has the modules \lstinline!tuple!, \lstinline!predicate! and
\lstinline!function!. If any of the corresponding types is used the module has
to appear directly or indirectly in the use block of the using module.



\section{Named features}

\noindent Syntax:
\begin{lstlisting}
    named_feature: feature_name ['(' formals ')' ]
                   [':' [ghost] type]
                   [feature_body]

    feature_name:  identifier
    |              '(' operator ')'
    |              number
    |              true
    |              false

    formals: {formal_argument ',' ...}+

    formal_argument:  identifier [':' type]

    feature_body:  '->' expression
    |              '='  expression
    |              [require_block]
                   [feature_implementation]
                   [ensure_block]
                   end

    require_block: require {expression separator ...}+

    ensure_block:  ensure  {expression separator ...}+

    feature_implementation:  deferred
\end{lstlisting}

\noindent Rules and validity:
\begin{enumerate}
\item In a named feature at least one of the optional components must be
  present.

\item In the feature body at least one of the optional components must be
  present. I.e. a feature body cannot consist of the keyword \lstinline!end! only.

\item If the feature body is \lstinline!-> expression! then formal arguments must be
  present and the expression must not have any preconditions i.e. the function
  is total.

\item If the feature body is \lstinline!= expression! then the feature is a constant and
  no formal arguments are allowed and the expression must not have any
  preconditions.

\item If the feature name is an operator there must be one formal argument for
  unary operators and two formal arguments for binary operators. Note that
  most of the operators can be unary and binary.

\item If the implementation of a function is not computable (e.g. if it
  involves universal or existential quantification) then the keyword
  \lstinline!ghost!  must be present in the result type.

\item The postcondition of a function might be of the form
  \lstinline!Result = exp!. If the expression contains non computable elements
  then the function must be declared as ghost function.

\item If the postcondition specifies only properties of the return value then
  the function has to be declared as ghost function and the proof engine must
  be able to prove the existence and uniqueness of the return value.
\end{enumerate}


\noindent Examples:

\noindent A simple declaration without body:
\begin{lstlisting}
    my_function (a,b:A, d:D): RT
\end{lstlisting}
%
This is a valid declaration of a function having the signature
\lstinline!(A,A,B):RT!. The function has no body. Within an implementation
file such a feature is practically useless because you cannot do anything with
it. In an interface file this declaration can be useful if assertions are
given which supply useful properties of the function.

\noindent A constant and a binary operator without body and a unary and binary operator
with body:
\begin{lstlisting}
    false: BOOLEAN

    (==>) (a,b:BOOLEAN): BOOLEAN

    (not) (a:BOOLEAN): BOOLEAN   -> a ==> false

    (=)   (a,b:BOOLEAN): BOOLEAN -> (a ==> b) and (b ==> a)
\end{lstlisting}

\noindent The empty set and the universal set defined as constants with body:
\begin{lstlisting}
    G: ANY

    0: G? = {x: false}
    1: G? = {x: true}
\end{lstlisting}


\noindent A ghost function involving non computable elements in its definition:
\begin{lstlisting}
    PO: PARTIAL_ORDER

    is_lower_bound (a:PO, p:PO?): ghost BOOLEAN
        -> all(x) x in p ==> a <= x
\end{lstlisting}


\noindent Declaration of a function with a precondition:
\begin{lstlisting}
    A: ANY; B: ANY

    value_at (f:A->B, a:A):B -> f(a)  -- ILLEGAL!!

    value_at (f:A->B, a:A):B
        require
            a in f.domain
        ensure
            Result = f(a)
        end
\end{lstlisting}
%
All functions in Albatross are potentially partial. Therefore the first
declaration is not valid because the expression has a precondition. The second
declaration is valid because it makes the precondition explicit.

\noindent Declaration of a function by using properties:
\begin{lstlisting}
    A:ANY; B:ANY

    inverse (f:A->B): ghost (B->A)
        require
            f.is_injective
        ensure
            Result.domain = f.range
            all(x) x in f.domain ==> Result(f(x)) = x
        end
\end{lstlisting}
%
The inverse of a function has to be declared as a ghost function because the
return value of \lstinline!inverse! is specified in the postcondition with
properties and not directly with a computable formula.

Note: If the return value of a function is given by properties then proof
engine must be able to prove that the return value exists and that it is
unique.



\section{Assertions} \label{assertions}

\noindent Syntax:
\begin{lstlisting}
    assertion_feature: all '(' formals ')'
                       [require_block]
                       [proof_block]
                       ensure_block
                       end

    proof_block:       proof {proof_expression separator ...}+

    proof_expression:  expression
    |                  assertion_feature
    |                  [require_block]
                       proof_block
                       ensure_block
                       end
\end{lstlisting}

\noindent Rules and validity:
\begin{enumerate}
\item All assertions must be provable by the proof engine.
\end{enumerate}

\noindent Examples:

\begin{lstlisting}
    all(p:G?)
        require
            some(x) x in p
        proof
            all(x) x in p ==> p /= 0    
        ensure
            p /= 0
        end  
\end{lstlisting}


\begin{lstlisting}
    all(p:G?)
        require
            p /= 0
        proof
            not (some(x) x in p) ==> false
        ensure
            some(x) x in p
        end
\end{lstlisting}

\begin{lstlisting}
    all(p,q:G?)
        require
             p /= 0
        proof
             some(x) x in p
             all(x)
                 require
                     p(x)
                 proof
                     x in p + q
                     some(x) x in {x: x in p + q}
                 ensure
                     p + q /= 0
                 end
        ensure
             p + q /= 0
        end
\end{lstlisting}



\section{Expressions}

\noindent Syntax:
\begin{lstlisting}
    expression: simple_expression
    |           operator_expression
    |           f_call
    |           oo_call
    |           tuple
    |           function_literal
    |           predicate_literal
    |           list
    |           listed_set
    |           quantified
    |           typed
    |           '(' expression ')'

    simple_expression:
                identifier | number | false | true | Result

    operator_expression:
                expression binary_operator expression
    |           unary_operator expresssion
    
    f_call:     expression '(' actuals ')'

    oo_call:    expression '.' identifier ['(' actuals ')']

    tuple:      '(' expression ',' {expression ',' ...}+ ')'

    actuals:    {expression ',' ...}+

    function_literal:
                '(' formals ')' [':' type] '->' expression
    |           agent '(' formals ')' [':' type]
                    [require_block]
                    ensure_block
                    end
                   
    predicate_literal:
                '{' formals ':' expression '}'

    listed_set: '{' {expression ',' ...}+ '}'

    list:       '[' {expression ',' ...}* ']'

    quantified: quantifier '(' formals ')' expression

    quantifier: all | some

    typed:      expression ':' type

    binary_operator:
                '+' | '-' | '*' | '/' | and | or | '==>' |
                '^' |
                '<=' | '<' | '>' | '>=' |
                '=' | '/=' | '~' | '/~' |
                free_operator

    unary_operator:
                not | old | '+' | '-' | '*' | free_operator
\end{lstlisting}

\noindent Precedence and associativity:

Ambiguities are resolved according to the following precedences and
associativities starting with the highest precedence.
\begin{lstlisting}
    '.'                        -- left
    not, old
    free_operator              -- left
    '^'                        -- right
    '*', '/'                   -- left
    '+', '-'                   -- left
    '<=', '<', '>=', '>', '=', '/=', '~', '/~', in, /in
    and, or
    '==>'                      -- right
\end{lstlisting}


\noindent Examples:

Functions can be called in functional or in object oriented notation. The
following expressions are equivalent

\begin{lstlisting}
    -- functional                    object oriented

    f(x,y,...)                       x.f(y,...)

    f(x)                             x.f

    f(f(f(f(x))))                    x.f.f.f.f
\end{lstlisting}

If the argument of a call is a tuple then nested parentheses are not
necessary, i.e. the following simplified notation is possible (only for
functional notation).

\begin{lstlisting}
    -- detailed                     simplified
    r((x,y))                        r(x,y)
\end{lstlisting}

Tuples are parsed with right association.
\begin{lstlisting}
    -- tuple                        parsed as
    (a,b,c,...)                     (a,(b,(c,...)))
\end{lstlisting}

The quantifiers and the arrow \lstinline!->! are greedy i.e. they span to the
right as far as possible. To limit the span parentheses can be used.

Types are greedy as well, i.e. they span to the left as far as possible.

A listed sets and lists are just a shorthand notations. The parser does the following
expansion

\begin{lstlisting}
     -- shorthand                   expansion
     {a,b,...}                      a.singleton + b.singleton + ...

     [a,b,...,z]                    a ^ b ^ ... ^ z ^ []
\end{lstlisting}



\section{Lexical Conventions} \label{lexconv}


There are two types of comments. The first one starts with the sequence
\lstinline!--! followed by a blank and spans to the end of the line. The
second one begins with the sequence \lstinline!{:! and ends with
  \lstinline!:}! and spans an abitrary amount of text. The second comment type
can be nested. Comments are completely ignored by the compiler.

\newcommand {\semicol}{\lstinline!;!}

A separator is either the semicolon or a newline between an endtoken and a
begintoken.

\begin{lstlisting}
    begintoken:    identifier
                   uidentifier
                   number
                   require
                   ensure
                   proof
                   true
                   false
                   not
                   '('
                   '['
                   '{'

    endtoken:      identifier
                   uidentifier
                   number
                   true
                   false
                   ')'
                   ']'
                   '}'
\end{lstlisting}
Note that no operator can be an endtoken because Albatross does not have
postfix operators. Furthermore nearly all operators cannot be starttokens
either. Therefore if a new expression starts with a unary operator the
expressions has to be either parenthesized or the previous expression has to
be terminated with a semicolon.

\begin{lstlisting}
    identifier:    [a-z][a-z0-9_]*

    uidentifier:   [A-Z][A-Z0-9_]*

    number:        [0-9]+

    free_operator: [+-/*<>=~|^]+
\end{lstlisting}

\noindent The following identifiers are reserved words in Albatross:
\begin{lstlisting}
    all          and          as          assert
    case         check        class       create
    deferred     do
    else         elseif       end         ensure
    false        feature      from
    ghost
    if           immutable    import      in          inherit
    inspect      invariant
    local
    not          note
    old          or
    proof
    redefine     rename       require
    some
    then         true
    undefine     use
    variant
    while
\end{lstlisting}


%%% Local Variables: 
%%% mode: latex
%%% TeX-master: "albatross"
%%% End: 

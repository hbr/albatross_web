\section{Important Modules of the Base Library}

\subsection{The Module 'any'}

The class \lstinline!ANY! declared in the module \lstinline!any! is the base
class for nearly all classes (therefore the name), because it introduces
equality.

The module \lstinline!any! uses some important concepts of Albatross
like deferred classes, type variables, inheritance, deferred functions and
assertions.

The interface file of the module \lstinline!any! contains just a handful of
declarations:

\begin{lstlisting}
    -- file: any.ali

    use boolean end

    deferred class ANY end

    G: ANY


    (=) (a,b:G): BOOLEAN    deferred end

    (/=) (a,b:G): BOOLEAN -> not (a = b)

    all(a:G) deferred ensure a = a end

    all(a:G) ensure a /= a ==> false end

    immutable class boolean.BOOLEAN
    inherit         ANY end
\end{lstlisting}

The class \lstinline!ANY! is a deferred class i.e. no objects of type
\lstinline!ANY! can be created. A deferred class is meant to be inherited by
other classes which implement the deferred features and assertions.

Only deferred classes can have deferred functions and assertions.

The second declaration is the declaration of the type variable
\lstinline!G!. The scope of a type variable is the module. A type variable has
a concept. The concept of the type variable \lstinline!G! is
\lstinline!ANY!. I.e. in any places where the type variable \lstinline!G!
occurs it can be replaced by any type which satisfies \lstinline!ANY!
i.e. which inherits from \lstinline!ANY!.

The module declares the two functions \lstinline!=! and \lstinline!/=!. The
function \lstinline!=! is a deferred function which has to be present in all
descendants of the class \lstinline!ANY!, either declared in the descendant
with a definition or redeclared as deferred. The function \lstinline!/=! has a
definition.

Furthermore the module declares a deferred assertion which states reflexivity
of equality. Like deferred functions all deferred assertions have to be
available in any descendant.

The second assertion is effective (i.e. not deferred) which states that the
implementation file has a proof for this statement. This assertion is
available to all descendants of \lstinline!ANY! for free.

The last statement is an inheritance statement. It states that the class
\lstinline!BOOLEAN! of the module \lstinline!boolean! can inherit
\lstinline!ANY! i.e. it has been successfully verified that the class
\lstinline!BOOLEAN! has a reflexive equality function and therefore is
entitled to inherit from \lstinline!ANY! the effective function \lstinline!/=!
and the effective assertion stating that irreflexivity implies falsehood.


\subsection{The Module 'predicate'}
\label{module-predicate}

The concept of a predicate is very important in Albatross. A predicate is a
total boolean valued function over a certain type. All objects which satisfy a
predicate form a set. Therefore a predicate represents a set.

Let us look into the interface file of the module \lstinline!predicate! to
explain some concepts.

\begin{lstlisting}
    -- file: predicate.ali

    use any end

    G: ANY

    immutable class PREDICATE[G] end

    (in)  (a:G, p:G?): BOOLEAN
    (/in) (a:G, p:G?): BOOLEAN -> not p(a)

    (<=) (p,q:G?): ghost BOOLEAN -> all(x) p(x) ==> q(x)
    (=)  (p,q:G?): ghost BOOLEAN -> p <= q and q <= p


    all(p:G?) ensure p = p end

    immutable class PREDICATE[G]
    inherit         ghost ANY end
\end{lstlisting}

The module \lstinline!predicate! uses the module \lstinline!any! and since the
module \lstinline!any! uses the module \lstinline!boolean! the module
\lstinline!predicate! uses the module \lstinline!boolean! implicitly.

For any type \lstinline!G! inheriting \lstinline!ANY! we can form the type
\lstinline!PREDICATE[G]!. Predicates are used frequently in
Albatross. Therefore the shorthand \lstinline!G?! has been introduced to be
equivalent to \lstinline!PREDICATE[G]!.

The expression \lstinline!p(x)! states that \lstinline!x! satisfies the
predicate \lstinline!p!. The expression \lstinline!x in p! is equivalent.

The function \lstinline!<=! defines the subset relation. A predicate/set
\lstinline!p! is a subset of \lstinline!q! if all elements of \lstinline!p!
occur in \lstinline!q! as well.

The function \lstinline!<=! has to be declared as a ghost function because it
uses a universally quantified expression in its definition.

Two sets are equal if they are mutually subsets. The equality function has to
be declared as ghost function as well because it uses another ghost function
in its definition.

The reflexivity of equality has been proved in the implementation file,
therefore the class \lstinline!PREDICATE! is entitled to inherit from the
class \lstinline!ANY!.

The inheritance relation has to be declared as ghost inheritance because the
equality function is present in the class \lstinline!PREDICATE! only as a
ghost function. As a consequence all functions inherited from the class
\lstinline!ANY! are inherited as ghost functions as well.

\begin{lstlisting}
    -- more declarations of the file 'predicate.ali'
    0: G? = {x: false}    -- empty set
    1: G? = {x: true}     -- universal set

    singleton (a:G): G? -> {x: x = a}  -- singleton set with one element
         -- {a} is a shorthand for a.singleton

    -- set union, intersection and difference
    (+)  (p,q:G?): G?  -> {x: p(x) or q(x)}
    (*)  (p,q:G?): G?  -> {x: p(x) and q(x)}
    (-)  (p,q:G?): G?  -> {x: p(x) and not q(x)}

    -- set complement
    (-)  (p:G?): G?    -> {x: not p(x)}

    -- union and intersection of a collection of sets
    (+)  (pp:G??): ghost G?  -> {x: some(p) pp(p) and p(x)}
    (*)  (pp:G??): ghost G?  -> {x: all(p)  pp(p) ==> p(x)}

    -- Leibniz rule
    all(a,b:G, p:G?)
        ensure
            a = b  ==>   p(a) ==>  p(b)
        end
\end{lstlisting}

The empty and the universal set over a type are declared as
constants. Constants are defined by the equality operator and not by the arrow
operator.

Albatross requires type annotations only at the top level and if the type
cannot be inferred from the context. The type of the bound variable
\lstinline!x! in \lstinline!{x: true}! can be inferred from the context
because the declared constant has type \lstinline!G?! therefore the variable
\lstinline!x! has to have the type \lstinline!G!.

An expression of the form \lstinline!{x: exp}! is a predicate expression and
the variable \lstinline!x! can occur in the expression \lstinline!exp!. This
predicate describes the set of all objects \lstinline!x! which satisfy the
expression \lstinline!exp!. The Albatross proof engine does beta reductions of
the form
\begin{lstlisting}
    a in {x: exp_x}    -- or equivalent {x: exp_x}(a)

    -- beta reduces to
    exp_a   -- all variables 'x' in 'exp_x' replaced by 'a'
\end{lstlisting}
in both forwards and backwards directions.

The function \lstinline!singleton! is the first non-operator function used in
this tutorial. It has one argument. The most natural way to call this function
is like \lstinline!singleton(a)!. This form is called the functional
notation. However Albatross has an equivalent object oriented notation with a
dot notation, extracting the first argument and placing it as the target object,
before the dot. Therefore the expression \lstinline!a.singleton! is an
equivalent function call.

In order to support a more mathematical notation the abbreviation
\lstinline!{a}! has been introduced. The language parser converts every
expression of the form \lstinline!{a,b,...}! to
\lstinline!a.singleton + b.singleton + ...!.

The following functions demonstrate that function names or operators do not
need to be unique in Albatross. Only the function name (operator name)
together with the signature has to be unique. Therefore \lstinline!-! can be
used as a binary operator for set difference and as a unary operator for set
complement.

The last declaration is the so called {\em Leibniz rule} of equality.  The
Leibniz rule expresses a concept which is very important in Albatross. If two
objects are equal there is no possibility to distinguish them. The Albatross
compiler enforces this rule by not allowing any definition of equality which
is not strong enough to satisfy the Leibniz rule.

The Leibniz rule allows you to do rewritings. As soon as there is an equality
of the form \lstinline!a = b! with specific \lstinline!a! and \lstinline!b! in
the context the proof engine does a partial specialization of the Leibniz rule
and adds
%
\begin{lstlisting}
    all(p) p(a) ==> p(b)
\end{lstlisting}
%
to the context. How can you trigger rewritings using this rule? If you can
prove the assertion \lstinline!exp_a! in which \lstinline!a! occurs as a
subexpression you can replace this subexpression by a fresh variable
\lstinline!x! to get the expression \lstinline!exp_x!. Now you can form the
predicate \lstinline!{x: exp_x}! and the subexpression \lstinline!a!
certainly satisfies this predicate, because you can prove
\lstinline!exp_a!. Now it is sufficient to state
%
\begin{lstlisting}
    b in {x: exp_x}
\end{lstlisting}
%
in a proof because the compiler recognizes \lstinline!all(p) p(a) ==> p(b)! as
a valid backward rule and tries to prove \lstinline!a in {x: exp_x}! which
succeeds. After the proof the proof engine forward closes the assertion
\lstinline!b in {x: exp_x}! by beta reduction and adds \lstinline!exp_b! to
the context which is \lstinline!exp_a! where the subexpression \lstinline!a!
has been substituted by the subexpression \lstinline!b!

%%% Local Variables: 
%%% mode: latex
%%% case-fold-search: nil
%%% TeX-master: "albatross"
%%% End: 

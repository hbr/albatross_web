\section{Using and Proving Abstractions}

In this chapter we show how to use deferred classes to create
abstractions. Abstractions are very useful because they allow us to define
functions based on some deferred functions and to prove assertions based on
some deferred assertions.

The inheritance mechanism of Albatross is a powerful reuse mechanism. All
classes which inherit from an abstraction get all functions defined in the
abstraction and all assertions proved in the abstraction for free.

We demonstrate the concepts by creating an abstraction for partial orders and
linear orders. The base library already has these abstractions. In this
chapter we create some simplified versions of the abstractions in the base
library. After having read this chapter it might be interesting to read the
versions in the base library.



\subsection{Partial Order}

In mathematics a partial order is a set with a binary relation which is
reflexive, antisymmetric and transitive.

It is easy to model such an abstraction in Albatross.

\begin{lstlisting}
    -- file: partial_order.al
    use alba.base.predicate end

    deferred class PARTIAL_ORDER end

    PO: PARTIAL_ORDER

    (=)  (a,b: PO): BOOLEAN  deferred end
    (<=) (a,b: PO): BOOLEAN  deferred end
    (<)  (a,b: PO): BOOLEAN  -> a <= b and a /= b
    (>=) (a,b: PO): BOOLEAN  -> b <= a
    (>)  (a,b: PO): BOOLEAN  -> b < a

    all(a,b,c:PO)
        deferred ensure
            a = a
            a <= a                             -- reflexivity
            a <= b  ==>  b <= a  ==>  a = b    -- antisymmetry
            a <= b  ==>  b <= c  ==>  a <= c   -- transitivity
        end

    deferred class PARTIAL_ORDER inherit ANY end
\end{lstlisting}

The transitivity of the \lstinline!>=! operator is trivial to prove
\begin{lstlisting}
    all(a,b,c:PO)
        require
            a >= b
            b >= c
        ensure
            a >= c
        end
\end{lstlisting}

Let's understand why this proof succeeds. After shifting the assumptions into
the context the proof engine forward closes the context. Forward closing
includes expansion of functions. Therefore after shifting the assumptions the
context consists of
\begin{lstlisting}
    a >= b;  b >= c             -- assumptions

    b <= a                      -- function expansion

    c <= b                      -- function expansion

    all(x) b <= x  ==> c <= x   -- partial specialization of transitivity

    ...
\end{lstlisting}

Now the proof engine tries to prove the goal. It cannot prove it
immediately. Therefore it searches for and generates backward
rules. Generation of backward rule includes function expansion. The expanded
goal is \lstinline!c <= a!. The expanded goal matches the conclusion of the
partial specialization and therefore backward reasoning can continue. The
substituted premise is already in the context and therefore the proof
succeeds.

This little proof is a nice demonstration how forward and backward reasoning
work hand in hand and that the proof engine can do many detailed steps which
would be very tedious to do them by hand.


The \lstinline!<! operator should be transitive as well. However the proof
engine fails to prove the transitivity without help.
%
\begin{lstlisting}
    all(a,b,c:PO)
        require
            a < b;  b < c
        ensure
            a < c        -- PROOF FAILS!
        end
\end{lstlisting}

Let's see the result of forward reasoning after shifting the assumptions
\begin{lstlisting}
    a <= b
    a /= b
    a = b ==> false
    b <= c
    b /= c
    b = c ==> false
    ...
\end{lstlisting}
Backward reasoning generates the following subgoals
\begin{lstlisting}
    require  a = c
    proof    ...
    ensure   false end

    a = c ==> false
    a /= c
    a <= c  -- proved by transitivity of '<='
    a < c
\end{lstlisting}

I.e. the proof can succeed if the assumption of \lstinline!a = c! leads to
falsehood. In order to prove falsehood the proof engine generates the subgoals
\begin{lstlisting}
    ...
    b <= c
    c <= b
    b = c
    false
\end{lstlisting}
Now it might be clear what the missing link is. The proof engine cannot prove
\lstinline!c <= b!. However we have \lstinline!a <= b! and \lstinline!a = c!
in the context. If we trigger a term rewrite of the first by the second one we
get \lstinline!c <= b! into the context and the proof succeeds. Term rewriting
can be triggered by applying the Leibniz rule (see chapter on the module
\lstinline!predicate!).

The following successfully proves the transitivity of \lstinline!<!.

\begin{lstlisting}
    all(a,b,c:PO)
        require
            a < b
            b < c
        proof
            require
                a = c
            proof
                c in {x: x <= b}
            ensure
                false
            end
        ensure
            a < c
        end
\end{lstlisting}
Note that the proof engine needs just a little hint on how to conclude
falsehood from the assumption \lstinline!a = c!. It is sufficient to provide
this hint. The detailed steps, which are numerous, are filled in by the proof
engine.

In the base library we have already an example of a class which satisfies the
requirements of a partial order. The subset relation defined in the module
\lstinline!predicate! is certainly reflexive, antisymmetric and transitive.

We can prove this conjecture by
\begin{lstlisting}
    G: ANY

    all(p,q,r:G?)
        ensure
            p <= p

            p <= q  ==>  q <= p  ==>  p = q

            p <= q  ==>  q <= r  ==>  p <= q
        end
\end{lstlisting}
By using function expansion and applying the basic techniques already seen in
this tutorial it is evident that the proof engine proves these claims
successfully.

Having reflexivity, antisymmetry and transitivity of the subset relation we
can state that the class \lstinline!PREDICATE! is entitled to inherit the
class \lstinline!PARTIAL_ORDER!.

\begin{lstlisting}
    immutable class
        predicate.PREDICATE[G]
    inherit
        ghost PARTIAL_ORDER
    end
\end{lstlisting}

Now the class \lstinline!PREDICATE! inherits all functions defined in this
module (\lstinline!<,>=,<!) and all assertions proved in this module.

But in order to see that the abstraction is really useful we need more {\em
  clients} which use it. In the next chapter we will see another class which
can inherit from a partial order.

But before using the module we have to create an interface file for it. We
don't show the interface file here because looking at the interface files
already shown here it should be clear what it has to contain.


\subsection{Linear Order}

A linear order is the special case of a partial order where all elements are
comparable.

\begin{lstlisting}
    -- file: linear_order.al
    deferred class LINEAR_ORDER end

    LO: LINEAR_ORDER

    (=)  (a,b:LO): BOOLEAN  deferred end
    (<=) (a,b:LO): BOOLEAN  deferred end

    all(a,b,c:LO)
        deferred ensure
            a = a

            a <= b  or  b <= a

            a <= b  ==>  b <= a  ==>  a = b

            a <= b  ==>  b <= c  ==>  a <= c
        end

    deferred class LINEAR_ORDER inherit ANY end
\end{lstlisting}

Note that we have not included reflexivity because reflexivity is a
consequence of these deferred assertions.

\begin{lstlisting}
    all(a:LO)
        proof
            a <= a  or  a <= a
        ensure
            a <= a
        end
\end{lstlisting}

Now we have enough evidence to state that a linear order is a special case of
a partial order which is stated as an inheritance relation in Albatross

\begin{lstlisting}
    deferred class
        LINEAR_ORDER
    inherit
        PARTIAL_ORDER
    end
\end{lstlisting}

In a partial order pairs of elements can be unrelated. In a linear order this
cannot happen. This is the content of the second deferred assertion.

Therefore if we know that \lstinline!not (a <= b)! holds we must be able to
conclude \lstinline!b < a!. In order to prove this conclusion we have to prove
\lstinline!b <= a! and \lstinline!b /= a!. The first one is a consequence of
the fact that pairs are always related by \lstinline!<=! and if one relation
does not hold then the other must hold. The second one can be proved by
showing the assumption \lstinline!b = a! leads to a contradiction.

\begin{lstlisting}
    all(a,b:LO)
        require
            not (a <= b)
        proof
            a <= b  or  b <= a

            require  b = a
            proof    a in {x: x <= b}
            ensure   false end
        ensure
            b < a
        end
\end{lstlisting}
%
Note that we used the Leibniz rule to rewrite the assertion \lstinline!b <= b!
to \lstinline!a <= b! which contradicts the assumption.

We want to prove as well that \lstinline!not (a < b)!  leads to
\lstinline!b <= a!. We use a proof by contradiction i.e. we assume that the
conclusion does not hold. From the last assertion we know that this leads to
\lstinline!a < b!  which contradicts the assumption.

\begin{lstlisting}
    all(a,b:LO)
        require
            not (a < b)
        proof
            require  not (b <= a)
            proof    a < b
            ensure   false end
        ensure
            b <= a
        end
\end{lstlisting}

In a linear order two different elements have to be related in the strict order
relation \lstinline!<!.

\begin{lstlisting}
    all(a,b:LO)
        require
            a /= b
        proof
            require
                not (a < b)   -- implies b <= a
            ensure
                b < a
            end
        ensure
            a < b  or  b < a
        end
\end{lstlisting}

%%% Local Variables: 
%%% mode: latex
%%% case-fold-search: nil
%%% TeX-master: "albatross"
%%% End: 

\chapter{Proof Engine}


\section{Rules of the Proof Engine}


The proof engine is based on a view very simple rules and knows only of
universal quantification, existential quantification and implication.

\subsection{Modus Ponens and Deduction Law}

The modus pones law says: Whenever the assertions
\begin{lstlisting}
    a

    a ==> b
\end{lstlisting}
are valid with arbitrary boolean expressions \lstinline!a! and \lstinline!b!
then we can conclude the validity of
%
\begin{lstlisting}
    b
\end{lstlisting}

The modus ponens law allows the proof engine to draw conclusion by following
implications.

The deduction law allows us to prove implications. It says: In order to prove
the implication
\begin{lstlisting}
    a ==> b
\end{lstlisting}
with arbitrary boolean expression \lstinline!a! and \lstinline!b! we can
assume \lstinline!a! and prove \lstinline!b!  under this assumption. I.e. the
two assertions
\begin{lstlisting}
    a ==> b

    require
        a
    proof
        ...
    ensure
        b
    end
\end{lstlisting}
are equivalent i.e. if we can prove the latter we have proved the former.

The deduction law can be generalized to arbitrary implication chains. I.e. the
proof engine is allowed to prove the chain

\begin{lstlisting}
    a ==> b ==> ... ==> z
\end{lstlisting}
%
by proving the assertion
\begin{lstlisting}
    require
        a; b; ...
    proof
        ...
    ensure
        z
    end
\end{lstlisting}


\subsection{Generalization of Variables}

Assume that we want to prove the assertion

\begin{lstlisting}
    all(x,y,...) exp
\end{lstlisting}

I.e. we have to prove that \lstinline!exp! is valid for all possible values of the
variables.

Such a statement can be proved by assuming arbitrary values for the
variables and proving the expression.

The proof engine proves a universally quantified statement by shifting the
variables into the context and then proving the expression. If this is
successful then the validity of the universally quantified assertions is
concluded.

Furthermore the proof engine knows that in can reorder the variables and
bubble up variables i.e. it considers the two following assertions equivalent
(with proper renamings of variables to avoid name clashes).

\begin{lstlisting}
    all(x,y) a ==> b ==> all(z) c

    all(y,x,z) a ==> b ==> c
\end{lstlisting}

The last rule is used by the proof engine to transform universally quantified
expressions to prenex normal form.



\subsection{Specialization of Variables}

If there is a proved statement of the form
\begin{lstlisting}
    all(x:X, y:Y,...) exp
\end{lstlisting}
%
the proof engine is allowed to generated valid asssertions by substituting the
variables \lstinline!x,y,...! by arbitrary expressions of consistent type.

The specialization can be partial if the expression \lstinline!exp! is an
implication where the first assumption contains only a part of the
variables. E.g. if we have the proved assertion

\begin{lstlisting}
    all(x,y) f(x) ==> g(x,y)
\end{lstlisting}
and the expression \lstinline!a! has a type compatible to the type of
\lstinline!x! then the proof engine can do a partial specialization and
conclude the assertion

\begin{lstlisting}
    f(a) ==> all(y) g(a,y)
\end{lstlisting}

If the first premise in the implication chain does not contain any of the
variables then partial specialization can be done as well. E.g. the assertion
\begin{lstlisting}
    all(a:BOOLEAN) false ==> a       -- ex falso quodlibet
\end{lstlisting}
can be partially specialized to
\begin{lstlisting}
    false ==> all(a:BOOLEAN) a
\end{lstlisting}


\noindent Example:
\begin{lstlisting}
    all(x,y,z) x or y ==> (x ==> z) ==> (y ==> z) ==> z
    a in p or a in q

    -- allows the partial specialization
    (a in p or a in q) ==>
    all(z) (a in p ==> z) ==> (a in q ==> z) ==> z
\end{lstlisting}


\subsection{Existencial Quantification}

The existentially quantified assertion
%
\begin{lstlisting}
    some(x,y,...) exp
\end{lstlisting}
%
can be proved by substituting the variables \lstinline!x,y,...! in the
expression \lstinline!exp! by some other expressions and proving the
substituted expression. This method is called {\em finding a witness for the
existentially quantified variables}.

If the above existentially quantified assertion has already been proved the
proof engine is allowed to conclude the following assertion from it.
\begin{lstlisting}
    all(a:BOOLEAN) (all(x,y,...) exp ==> a) ==> a
\end{lstlisting}
The validity of this assertion should be obvious. If there exist some values
for the variables \lstinline!x,y,...! so that the expression \lstinline!exp!
is valid and for all values of the variables the validity of \lstinline!exp!
implies the validity of \lstinline!a! then the assertion \lstinline!a! is
valid.



\subsection{Function Expansion}

The proof engine is allowed to expand function definitions. E.g. boolean
negation is defined as
%
\begin{lstlisting}
    (not) (a:BOOLEAN) ->  a ==> false
\end{lstlisting}
%
Therefore the proof engine can expand the expression
%
\begin{lstlisting}
    not (a <= b)
\end{lstlisting}
%
to
%
\begin{lstlisting}
    a <= b  ==> false
\end{lstlisting}


\subsection{Beta Reduction}

Albatross has two kind of lambda expressions: function and predicate literals
of the form
%
\begin{lstlisting}
    (x,y,...) -> exp

    {x,y,...: exp}
\end{lstlisting}
%
The following expressions can be substituted by \lstinline!exp2! where
\lstinline!exp2! is \lstinline!exp!  with all variables \lstinline!x,y,...!
substituted by the expressions \lstinline!a,b,...!.
%
\begin{lstlisting}
    ((x,y,...) -> exp) (a,b,...)

    {x,y,...: exp} (a,b,...)

    (a,b,...) in {x,y,...: exp}
\end{lstlisting}
Note that \lstinline!a in p! and \lstinline!p(a)! are equivalent for
predicates \lstinline!p!.



\section{How the Proof Engine Works}

A context of the proof engine is a sequence of variables and a sequence of
assumptions and a sequence of proved assertions. Contexts can be stacked. A
new inner context is opened by shifting variables and assumptions into the
context. Within the context assumptions are treated as proved statements. By
leaving a context the variables and assumptions have to be discharged.

\subsection{Proof of an Assertion Feature}

An assertion feature has the general form:
\begin{lstlisting}
    all(a,b,...)
        require
            assumption_1
            assumption_2
            ...
        proof
            intermediate_1
            intermediate_2
            ...
        ensure
            conclusion_1
            conclusion_2
        end
            
\end{lstlisting}
where all assumptions and conclusions are expressions and all intermediate
assertions are expressions or nested assertion features with or without new
variables.

The proof engine proves an assertion feature by the following
algorithm.

\begin{enumerate}
\item Open an new context

\item Shift all variables into the context.

\item For all assumptions: Prove the preconditions of the assumption and shift
  the assumtion into the context.

\item For all intermediate assertions:
  \begin{enumerate}
  \item If it is an expression then prove all preconditions and prove the
    expression and add it to the context.
  \item If it is an assertion feature then apply this algorithm recursivly and
    add the result to the context.
  \end{enumerate}

\item For all conclusions prove the preconditions of the conclusion and the
  conclusion and add the conclusion to the context.

\item Leave the context and add all conclusions with properly discharged
  variables and assumtions to the outer context
\end{enumerate}


\subsection{Prove an Assertion}

Proving of any assertion (precondition or assertion expression) is done by
the following algorithm:

\begin{enumerate}
\item Forward close the context by forward reasoning.

\item Enter the assertion as deep as possible i.e. split universal
  quantification and add the corresponding variables and split implications
  and add assumptions.

\item Directyl prove the goal. In case of success the proof is done. In case
  of failure continue.

\item Prove the goal by backward reasoning.

\item Reverse the entering process and discharge the proved goal.
\end{enumerate}


\subsection{Directly Prove a Goal}

The proof engine tries to directly prove a goal after it has been fully
entered i.e. the goal is neither a universally quantified expression nor an
implication.

A goal is directly provable if one of the following holds:
\begin{enumerate}
\item The goal is already in the context.

\item A schematic assertion can be found in the context together with a
  substitution so that the substitution applied to the schematic assertion
  makes it identical with the goal.

\item The goal is an existentially quantified assertion and a witness
  expression can be found in the context to prove the existential
  quantification.

\item The goal is an equality of the form
  \begin{lstlisting}
      f(a1,a2,...) = f(b1,b2,...)
  \end{lstlisting}
  and the assertions
  \begin{lstlisting}
      a1 = b1
      a2 = b2
      ...
  \end{lstlisting}
  are already in the context.
\end{enumerate}



\subsection{Entering and Discharging}

An assertion expression has the general form
\begin{lstlisting}
    all(x,y,...) a ==> b ==> ... ==> z
\end{lstlisting}
where the universal quantification and the premises of the implication chain
are optional. If no universal quantification and no premises are present then
the assertion expression is the unsplittable expression \lstinline!z! and the
enter and the discharge procedures have nothing to do.

The entering procedure consists in
\begin{enumerate}
\item Shift the variables \lstinline!x,y,...! into the context.

\item Shift the assumptions \lstinline!a,b,...! one by one into the context.

\item Forward close the context by forward reasoning.

\item If the target \lstinline!z! can be entered apply the procedure recursively
\end{enumerate}

After entering the assertion expression there is a goal \lstinline!z! to be
proved which cannot be entered any more. The goal (or a witness for it) can
either be found in the context or can be proved by backward reasoning (or the
proof fails).

Lets assume that the prove of the target goal \lstinline!z! has
succeeded. Then the entering process has to be reversed by the discharge
procedure. Since the entering procedure is recursive we have to consider the
general case that the original assertion expression has the form
%
\begin{lstlisting}
    all(x,y,...) a ==> b ==> ... ==> all(t,u,...) z0
\end{lstlisting}
%
The proof engine normalizes the original assertion to
%
\begin{lstlisting}
    all(x,y,...,t,u,...) a ==> b ==> ... ==> z0
\end{lstlisting}
%
with proper renaming to avoid name clashes. Furthermore all unused variables
are eliminated and the remaining variables are sorted to reflect the order in
which they appear in the implication chain.


\subsection{Forward Reasoning}

Forward reasoning is the process of drawing conclusions from the assertions
within a context. A context is forward closed by applying the following rules
as long as possible:
%
\begin{enumerate}
\item Find a pair of assertions of the form \lstinline!a! and
  \lstinline!a ==> b! and apply the moduls ponens law by adding \lstinline!b!
  to the context.

\item Find a pair of assertions of the form \lstinline!a! and
  \lstinline!all(x,y,...) p ==> ...!  where \lstinline!p! is not a pure
  variable and a (partial) substitution of the variables so that \lstinline!p!
  becomes identical to \lstinline!a! if the substitution is applied. Add the
  (partially) specialized assertion to the context.

\item Find an existentially quantified assertion \lstinline!some(x,y,...) exp!
  and add the assertion
  \begin{lstlisting}
      all(a) (all(x,y,...) exp) ==> a
  \end{lstlisting}
  to the context.

\item Find an assertion where an evaluation is possible. Add the evaluated
  assertion to the context.
\end{enumerate}

\noindent Some restrictions apply to the application of the first and the second rule.
\begin{enumerate}
\item The (partial) specialization of a schematic assertion of the form
  \begin{lstlisting}
      all(x,y,...) a ==> b ==> ... ==> z
  \end{lstlisting}
  is called intermediate and remains intermediate until the final conclusion
  \lstinline!z! has been added fully specialized to the context.

\item An intermediate assertion is never used as the premise in the modus
  ponens rule.

\item The final fully specialized conclusion of an intermediate assertion is
  added to the context only if the conclusion is simpler than all premises. A
  term \lstinline!a! is simpler than a term \lstinline!b! if its expression
  tree has less nodes than the expression tree of the term \lstinline!b! after
  all function expansions have been done.
\end{enumerate}
These restrictions guarantee the termination of forward reasoning.


\subsection{Backward Reasoning}

If an assertion has been fully entered and is not directly provable then the
proof engine starts to prove the goal by backward reasoning.

\begin{enumerate}
\item Find and generate backward rules. A valid backward rule is an implication
  chain of the form
  \begin{lstlisting}
      p1 ==> p2 ==> ... ==> goal
  \end{lstlisting}
  which has not yet been used in the proof.

\item Find one backward rule for which all premises can be proved. In case of
  success the proof is done by applying the modus ponens rule, in case of
  failure the proof fails.
\end{enumerate}

\noindent There are two ways to generate backward rules:
%
\begin{enumerate}
\item Find a schematic assertion of the form
  \begin{lstlisting}
      all(x,y,...) p1 ==> p2 ==> ... ==> z
  \end{lstlisting}
  where \lstinline!z! is not a variable and can be unified with the goal by
  some substitution of the variables. The fully specialized assertion is a
  valid backward rule. Restriction: If some of the premises (after the
  substitution and with all functions expanded) are more complicated than the
  goal (with all functions expanded) and the schematic rule has already been
  used before in the proof, then the generated backward rule is not valid.

\item If \lstinline!eval! is a valid evaluation of the goal then
  \begin{lstlisting}
      eval ==> goal
  \end{lstlisting}
  is a valid backward rule.
\end{enumerate}




\subsection{Evaluation}

Universally quantified expressions, implications and conjunctions
i.e. expressions of the following forms are never evaluated.
\begin{lstlisting}
    all(x,y,...) exp

    a and b

    a ==> b
\end{lstlisting}

Ways to evaluate an expression:
\begin{enumerate}
\item Expansion of the toplevel function. If the toplevel function is a
  function term then do a toplevel expansion of the function term.

\item If the toplevel function does not have a definition and the expression
  is neither universally quantified nor an implication nor a conjunction then
  the expression with all functions expanded recursively is a valid evaluation.

\item Beta reduction of the toplevel term.

\item If the expression has the subterms \lstinline!a1,a2,...! i.e. it has the
  form \lstinline!f(a1,a2,...)! and there are the assertions
  \begin{lstlisting}
      a1 = b2
      a2 = b2
      ...
  \end{lstlisting}
  in the context where the right hand side of the equation is simpler than the
  left hand side (after full function expansion), then
  \lstinline!f(b1,b2,...)! is a valid evaluation of the expression.
\end{enumerate}

%%% Local Variables: 
%%% mode: latex
%%% TeX-master: "albatross"
%%% End: 
